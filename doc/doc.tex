\documentclass{article}
\documentclass[11pt, a4paper, titlepage]{article}

% Set document dimensions
\usepackage[paper=a4paper,top=1cm,left=2cm,right=2cm,bottom=2cm, includefoot]{geometry}
\usepackage{float}
\usepackage[final]{pdfpages} % inludesvg

% Czech fonts
\usepackage[T1]{fontenc}
\usepackage[utf8]{inputenc}
\usepackage[czech]{babel}
\usepackage{amsmath}
\usepackage{listings}
\usepackage{wrapfig}
% \usepackage{titlesec}
\usepackage{color} % barvy
\usepackage{xcolor} % vytváření barev
\usepackage{multicol}
\usepackage{newverbs}
\usepackage{lipsum}

%\usepackage{fancyhdr}

\usepackage{fancyhdr}


\lstnewenvironment{code}[1][]%
{\noindent\minipage{\linewidth}\lstset{frameround=fttf,#1}}%
{\endminipage}%

\definecolor{codeprimary}{HTML}{3300CC}
\colorlet{keywordstyle}{codeprimary!50!black}
\lstset{
    language=C++,
    frameround=fftf,
    breaklines=true,
    keywordstyle=\color{keywordstyle}\ttfamily,
    basicstyle=\color{codeprimary},
    numberstyle=\color{black},
    backgroundcolor=\color{white},
    frame=single,
    tabsize=4,
    breaklines=true,
    captionpos=t,
    xleftmargin=\dimexpr\fboxsep+1pt\relax,
    xrightmargin=\fboxsep,
    numbers=none,
    showstringspaces=false,
    escapeinside={\#!}{\^^M},
    belowcaptionskip=0pt,
    belowskip=0pt,
    aboveskip=0pt,
}

\makeatletter
\newcommand\ic[1][green]{%
\@testopt{\@ic{#1}}{-#1}% Handle second optional argument
}
\def\@ic#1[#2]{%
\Collectverb{\@@ic{#1}{#2}}%
}
\def\@@ic#1#2#3{%
{\lstinline[basicstyle=\ttfamily\color{codeprimary},breaklines=true]|#3|}%
}
\newcommand{\icmacro}[1]{{\lstinline[basicstyle=\ttfamily\color{codeprimary},breaklines=true]|#1|}}
\makeatother


\setlength{\headheight}{3em}
\newcommand{\subsectionbreak}{\clearpage}

\begin{document}
    %%%%%%%%%%%%%%%%%%%%%%%%%%%%%%%%%%%%%%%%%%%%%%%%%%%%%%%%%%%%%%%%%%
% Dokumentační část projektu do IMS.
% VUT FIT
% @author Josef Kolář, xkolar71
% @author Iva Kavánková, xkavan05
% @date 2018; 05; 12
%%%%%%%%%%%%%%%%%%%%%%%%%%%%%%%%%%%%%%%%%%%%%%%%%%%%%%%%%%%%%%%%%%

\begin{titlepage}
    % \newgeometry{top=1in,top=2cm,left=2cm,right=2cm,bottom=2cm}

    \centering

    {\fontsize{20pt}{15pt}\bfseries
    VYSOKÉ UČENÍ TECHNICKÉ V~BRNĚ\\
    \vspace{8pt}
    Fakulta informačních technologií
    }

    \includegraphics[scale=0.7]{./assets/fit-logo.pdf}

    \vspace{22pt}

    {\Large Modelování a simulace\\}
    \vspace{4pt}
    {\LARGE \bfseries Chov hmyzu pro potravinářské a průmyslové účely}

    \vspace{180pt}
    {\Large \today}

    \vspace{90pt}
    {\Large \bfseries Autoři\\}
    \vspace{12pt}

    \begin{tabular}{ l c r }
        Josef Kolář & \texttt{xkolar71} \\
        Iva Kavánková & \texttt{xkavan05} \\
    \end{tabular}\\

\end{titlepage}
    \pagestyle{fancy}
    \lfoot{\emph{VUT FIT - IMS}}
    \rfoot{\emph{Josef Kolář, Iva Kavánková}}
    \rhead{...}
    \lhead{Modelování a simulace}
    \tableofcontents
    \pagebreak
    %\noindent\makebox[\linewidth]{\rule{\textwidth}{0.4pt}}

    \section{Úvod}
    Tato práce vznikla jako projekt do předmětu Modelování a simulace. Zabývá se simulací (viz [\ref{ims}], slajd č. 8)
    modelu (viz [\ref{ims}], slajd č. 7) chovu cvrčků banánových (dále jen crvčci). Na základě daného modelu a sady simulačních experimentů (viz [\ref{ims}], slajd č. 33),
    bude ukázáno chování systému v rozdílných podmínkách. Smyslem projektu je demonstrovat, smysluplnost a finanční efektivitu hypotetické farmy na cvrčky.

    \subsection{Zdroje informací}
    Autoři projektu jsou Josef Kolář a Iva Kavánková. Při tvorbě došlo k čeprání znalostí z přednášek předmětu Modelování a simulace, z odborné literatury
    [\ref{kniha}] a obzvlášť od paní inženýrky Olgy Kavánkové, která je středoškolskou pedagožkou s aprobací na biologii
    a matematiku, čímž ji patří velké poděkování. Díky jejím odborným znalostem byl vytvořen odpovídající abstraktní model
    (viz [\ref{ims}], slajd č. 9).

    \subsection{Ověřování validity modelu}
    Validita (viz [\ref{ims}], slajd č. 37) byla ověřována při postupném testování.


    \\
Smysl této povinné osnovy je dát praktický návod pro sepsání technické
zprávy k projektu. Číslované kapitoly jsou \textbf{naprosto závazné} -
jejich případná absence bude hodnocena bodovou srážkou. Berte technickou
simulační zprávu jako formu protokolu. Rozhodně se nejedná o beletrii,
váš osobní příběh nebo filosofickou úvahu - tyto formy si můžete dovolit
až v pokročilejším věku (kdy zřejmě už pochopíte, že beztak v technice
jsou preferována relevantní fakta, specificky u platících
zákazníků).\\[2\baselineskip]Kapitoly první úrovně mají závazný název.
Kapitoly druhé úrovně nemají závazný název, upravte si název dle
potřeby. V dokumentu jsou povinné tématické bloky kapitol první a druhé
úrovně s nenulovým obsahem. Kapitola první úrovně předpokládá určitý
text (uvedeno v popisu) rozšířený o témata kapitol druhé
úrovně.\\[2\baselineskip]\textbf{Podstatné je, že technická zpráva se
neobhajuje a nekomentuje. Všecko relevantní má být obsaženo v ní. Nejsou
přípustná žádná dodatečná ústní sdělení. Pokud čtenář z vaší technické
zprávy něco chybně pochopí, je to vždy vaše vina.}\\[2\baselineskip]Je
jasné, že různá témata budou klást důraz na jiné partie zprávy:\\

\begin{itemize}
\tightlist
\item
  implementační - literatura, koncepce, zpracování implementace, ověření
  na průkazném demopříkladě
\item
  modelační/aplikační - literatura a zjišťování v terénu, zdůvodnění
  koncepce modelu, simulační studie vedoucí k jasnému závěru
\end{itemize}

Bez ohledu na charakter tématu lze v technické dokumentaci VŽDY odděleně
popsat:\\

\begin{itemize}
\tightlist
\item
  Úvod a motivaci
\item
  Shrnutí relevantních faktů, zdroje informací
\item
  Koncepci metody, přístupu, modelu
\item
  Implementaci metody, modelu
\item
  Experimentování
\item
  Závěr
\end{itemize}

\subsubsection{1. Úvod}\label{uxfavod}

Úvod musí vysvětlit, \textbf{proč se celá práce dělá a proč má uživatel
výsledků váš dokument číst} (prosím, projekt sice děláte pro získání
zápočtu v IMS, ale mohou existovat i jiné důvody). Případně, co se při
čtení dozví.\\[2\baselineskip]Například:\\

\begin{itemize}
\tightlist
\item
  v této práci je řešena implementace \ldots{}, která bude použita pro
  sestavení modelu \ldots{}
\item
  na základě modelu a simulačních experimentů bude ukázáno chování
  systému \ldots{} v podmínkách \ldots{}
\item
  smyslem experimentů je demonstrovat, že pokud by \ldots{}, pak by
  \ldots{}
\item
  Poznámka: u vyžádaných zpráv se může uvést, že zpráva vznikla na
  základě požadavku \ldots{} (u školní práce takto zdůvod'novat projekt
  ovšem nelze, že). Je velmi praktické zdůvodnit, \textbf{v čem je práce
  náročná a proto přínos autora nepopiratelný} (např. pro zpracování
  modelu bylo nutno nastudovat \ldots{}, zpracovat, \ldots{} model je ve
  svém oboru zajímavý/ojedinělý v \ldots{}).
\end{itemize}

Podkapitoly:\\

\begin{itemize}
\tightlist
\item
  kapitola 1.1: Kdo se na práci podílel jako autor, odborný konzultant,
  dodavatel odborných faktů, význačné zdroje literatury/fakt, \ldots{}

  \begin{itemize}
  \tightlist
  \item
    je ideální, pokud jste vaši koncepci konzultovali s nějakou
    autoritou v oboru (v IMS projektu to je hodnoceno, ovšem není
    vyžadováno)
  \item
    pokud nebudete mít odborného konzultanta, nevadí. Nelze ovšem
    tvrdit, že jste celé dílo vymysleli s nulovou interakcí s okolím a
    literaturou.
  \end{itemize}
\item
  kapitola 1.2: V jakém prostředí a za jakých podmínek probíhalo
  experimentální ověřování validity modelu -- pokud čtenář/zadavatel
  vaší zprávy neuvěří ve validitu vašeho modelu, \textbf{obvykle vaši
  práci odmítne už v tomto okamžiku}.
\end{itemize}

\subsubsection{2. Rozbor tématu a použitých
metod/technologií}\label{rozbor-tuxe9matu-a-pouux17eituxfdch-metodtechnologiuxed}

Shrnutí všech podstatných faktů, které se týkají zkoumaného systému (co
možná nejvěcnějším a technickým (ideálně formálně matematickým)
přístupem, žádné literární příběhy). \textbf{Podstatná fakta o systému
musí být zdůvodněna a podepřena důvěryhodným zdrojem} (vědecký článek,
kniha, osobní měření a zjišťování). Pokud budete tvrdit, že ovce na
pastvě sežere dvě kila trávy za den, musí existovat jiný (důvěryhodný)
zdroj, který to potvrdí. Toto shrnutí určuje míru důvěryhodnosti vašeho
modelu (nikdo nechce výsledky z nedůvěryhodného modelu). Pokud nebudou
uvedeny zdroje faktů o vašem systému, hrozí ztráta bodů.\\

\begin{itemize}
\tightlist
\item
  kapitola 2.1: Popis \textbf{použitých postupů} pro vytvoření modelu a
  zdůvodnění, \textbf{proč jsou pro zadaný problém vhodné}. Zdůvodnění
  může být podpořeno ukázáním alternativního přístupu a srovnáním s tím
  vaším. Čtenář musí mít jistotu, že zvolené
  nástroje/postupy/technologie jsou ideální pro řešení zadaného problému
  (ovšem, ``dělám to v Javě, protože momentálně Java frčí\ldots{}''
  nemusí zadavatele studie uspokojit).
\item
  kapitola 2.2: Popis \textbf{původu} použitých metod/technologií (odkud
  byly získány (odkazy), zda-li jsou vytvořeny autorem) - převzaté části
  dokumentovat (specificky, pokud k nim přísluší nějaké autorské
  oprávnění/licence). Zdůvodnit potřebu vytvoření vlastních
  metod/nástrojů/algoritmů. Ve většině případů budete přebírat již
  navržené metody/algoritmy/definice/nástroje a je to pro školní projekt
  typické. Stejně tak je typické, že \textbf{studenti chybně vymýšlí již
  hotové věci a dojdou k naprostému nesmyslu} - je třeba toto nebezpečí
  eleminovat v tomto zdůvodnění.
\end{itemize}

\textbf{Velmi důležité, až fanaticky povinné, kontrolované a hodnocené:
na každém místě v textu, kde se poprvé objeví pojem z předmětu IMS bude
v závorce uveden odkaz na předmět a číslo slajdu, na kterém je pojem
definován}. Pokud bude významný pojem z předmětu IMS takto
nedokumentován v textu a zjevně bude používán nevhodným nebo nepřesným
způsobem, bude tento fakt hodnocen bodovou ztrátou. Tento požadavek je
míněn s naprostou vážností. Cílem je vyhnout se studentské tvůrčí
činnosti ve vysvětlování známých pojmů, což mnohdy vede k naprostým
bludům, ztrátě bodů a zápočtů. Pokud student pojem cituje korektně a
přesto nekorektně používá, bude to hodnoceno dvojnásobnou bodovou
ztrátou.\\[2\baselineskip]

\subsubsection{3. Koncepce - modelářská
témata}\label{koncepce---modeluxe1ux159skuxe1-tuxe9mata}

Konceptuální model je abstrakce reality a redukce reality na soubor
relevantních faktů pro sestavení simulačního modelu. Předpokládáme, že
model bude obsahovat fakta z ``Rozboru tématu''. Pokud jsou některá
vyřazena nebo modifikována, je nuto to zde zdůvodnit (například:
zkoumaný subjekt vykazuje v jednom procentu případů toto chování, ovšem
pro potřeby modelu je to naprosto marginální a smíme to zanedbat, neboť
\ldots{}). \textbf{Pokud některé partie reality zanedbáváte nebo
zjednodušujete, musí to být zdůvodněno a v ideálním případě musí být
prokázáno, že to neovlivní validitu modelu.} Cílem konceptuálního
(abstraktního) modelu je formalizovat relevantní fakta o modelovaném
systému a jejich vazby. Podle koncept. modelu by měl být každý schopen
sestavit simulační model.\\

\begin{itemize}
\tightlist
\item
  kapitola 3.1: Způsob vyjádření konceptuálního modelu musí být
  zdůvodněn (na obrázku xxx je uvedeno schéma systému, v rovnicích xx-yy
  jsou popsány vazby mezi \ldots{}, způsob synchronizace procesů je na
  obrázku xxx s Petriho sítí).
\item
  kapitola 3.2: Formy konceptuálního modelu (důraz je kladen na
  srozumitelnost sdělení). Podle potřeby uveďte konkrétní relevantní:

  \begin{itemize}
  \tightlist
  \item
    obrázek/náčrt/schéma/mapa (možno čitelně rukou)
  \item
    matematické rovnice - u některých témat (např. se spojitými prvky,
    optimalizace, \ldots{}) naprosto nezbytné. \textbf{Dobré je chápat,
    že veličiny (fyzikální, technické, ekonomické) mají jednotky, bez
    kterých údaj nedává smysl.}
  \item
    stavový diagram (konečný automat) nebo Petriho síť - spíše na
    abstraktní úrovni. Petriho síť nemá zobrazovat výpočty a přílišné
    detaily. Pokud se pohodlně nevejde na obrazovku, je nepoužitelná.
    Možno rozdělit na bloky se zajímavými detaily a prezentovat odděleně
    abstraktní celek a podrobně specifikované bloky (hierarchický
    přístup).
  \end{itemize}
\end{itemize}

\subsubsection{3. Koncepce - implementační
témata}\label{koncepce---implementaux10dnuxed-tuxe9mata}

Popište abstraktně architekturu vašeho programu, princip jeho činnosti,
významné datové struktury a podobně. Smyslem této kapitoly je podat
informaci o programu bez použití názvů tříd, funkcí a podobně. Tuto
kapitolu by měl pochopit každý technik i bez informatického vzdělání.
Vyjadřovacími prostředky jsou vývojové diagramy, schémata, vzorce,
algoritmy v pseudokódu a podobně. Musí zde být vysvětlena nosná myšlenka
vašeho přístupu.\\[3\baselineskip]

\subsubsection{4. Architektura simulačního
modelu/simulátoru}\label{architektura-simulaux10dnuxedho-modelusimuluxe1toru}

Tato kapitola má různou důležitost pro různé typy zadání. U
implementačních témat lze tady očekávat jádro dokumentace. Zde můžete
využít zajímavého prvku ve vašem simulačním modelu a tady ho
``prodat''.\\

\begin{itemize}
\tightlist
\item
  kapitola 4.1: Minimálně je nutno ukázat mapování abstraktního
  (koncept.) modelu do simulačního (resp. simulátoru). Např. které třídy
  odpovídají kterým procesům/veličinám a podobně.
\end{itemize}

\subsubsection{5. Podstata simulačních experimentů a jejich
průběh}\label{podstata-simulaux10dnuxedch-experimentux16f-a-jejich-prux16fbux11bh}

\textbf{Nezaměňujte pojmy testování a experimentování (důvod pro bodovou
ztrátu)!!!}\\
Zopakovat/shrnout \textbf{co přesně chcete zjistit experimentováním} a
proč k tomu potřebujete model. \textbf{Pokud experimentování nemá cíl,
je celý projekt špatně.} Je celkem přípustné u experimentu odhalit chybu
v modelu, kterou na základě experimentu opravíte. Pokud se v některém
experimentu nechová model podle očekávání, je nutné tento experiment
důkladně prověřit a chování modelu zdůvodnit (je to součást simulačnické
profese). Pokud model pro některé vstupy nemá důvěryhodné výsledky, je
nutné to zdokumentovat. Pochopitelně model musí mít důvěryhodné výsledky
pro většinu myslitelných vstupů.\\

\begin{itemize}
\tightlist
\item
  kapitola 5.1: Naznačit postup experimentování -- jakým způsobem
  hodláte prostřednictvím experimentů dojít ke svému cíli (v některých
  situacích je přípustné ``to zkoušet tak dlouho až to vyjde'', ale i ty
  musí mít nějaký organizovaný postup).
\item
  kapitola 5.2: Dokumentace jednotlivých experimentů - souhrn vstupních
  podmínek a podmínek běhu simulace, komentovaný výpis výsledků, závěr
  experimentu a plán pro další experiment (např. v experimentu 341. jsem
  nastavil vstup x na hodnotu X, která je typická pro \ldots{} a vstup y
  na Y, protože chci zjistit chování systému v prostředi \ldots{} Po
  skončení běhu simulace byly získány tyto výsledky \ldots{}, kde je
  nejzajímavější hodnota sledovaných veličin a,b,c které se chovaly
  podle předpokladu a veličin d,e,f které ne. Lze z toho usoudit, že v
  modelu není správně implementováno chování v podmínkách \ldots{} a
  proto v následujících experimentech budu vycházet z modifikovaného
  modelu verze \ldots{} Nebo výsledky ukazují, že systém v těchto
  podmínkách vykazuje značnou citlivost na parametr x \ldots{} a proto
  bude dobré v dalších experimentech přesně prověřit chování systému na
  parametr x v intervalu hodnot \ldots{} až \ldots{})
\item
  kapitola 5.3: Závěry experimentů -- bylo provedeno N experimentů v
  těchto situacích \ldots{} V průběhu experimentování byla odstraněna
  \ldots{} chyba v modelu. Z experimentů lze odvodit chování systémů s
  dostatečnou věrohodností a experimentální prověřování těchto \ldots{}
  situací již napřinese další výsledky, neboť \ldots{}
\end{itemize}

\subsubsection{6. Shrnutí simulačních experimentů a
závěr}\label{shrnutuxed-simulaux10dnuxedch-experimentux16f-a-zuxe1vux11br}

Závěrem dokumentace se rozumí \textbf{zhodnocení simulační studie a
zhodnocení experimentů} (např. Z výsledků experimentů vyplývá, že
\ldots{} při předpokladu, že \ldots{} Validita modelu byla ověřena
\ldots{} V rámci projektu vznikl nástroj \ldots{}, který vychází z
\ldots{} a byl implementován v \ldots{}).\\

\begin{itemize}
\tightlist
\item
  do závěru se nehodí psát poznámky osobního charakteru (např. práce na
  projektu mě bavila/nebavila, \ldots{}). \textbf{Technická zpráva není
  osobní příběh autora.}
\item
  absolutně nikoho nezajímá, kolik úsilí jste projektu věnovali,
  důležitá je pouze kvalita zpracování simulátoru/modelu a obsažnost
  simulační studie (rozhodně ne např.: projekt jsem dělal \ldots{}
  hodin, což je víc než zadání předpokládalo. Program má \ldots{} řádků
  kódu). \textbf{Pokud zdůrazňujete, že jste práci dělali významně déle
  než se čekalo, pak tím pouze demonstrujete vlastní neschopnost (to
  platí zejména v profesním životě).}
\item
  do závěru se velmi nehodí psát ``auto-zhodnocení'' kvality práce, to
  je výhradně na recenzentovi/hodnotiteli (např. v projektu jsem zcela
  splnil zadání a domnívám se, že můj model je bezchybný a výsledky
  taktéž). Statisticky častý je pravý opak autorova auto-zhodnocení.
  Pokud přesto chcete vyzdvihnout kvalitu svého díla (což je dobře), tak
  vaše výroky musí být naprosto nepopiratelně zdůvodněny a prokázány
  (např. pomocí jiného referenčního přístupu, matematického důkazu,
  analýzy, \ldots{}).
\end{itemize}

\subsubsection{Obecné poznámky}\label{obecnuxe9-poznuxe1mky}

\begin{itemize}
\tightlist
\item
  v dokumentaci neocením ``úvodní stránku s logem fakulty'' a obsah.
  Obvzláště obsah jako jedna stránka u čtyř-pěti stránkového dokumentu
  působí směšně. Tyto dvě strany pochopitelně smíte zprávě předřadit,
  nepočítám je ovšem do počtu stran zprávy.
\item
  grafy mají své náležitosti - identifikační název grafu (případně jeho
  číslo), cejchované osy s názvem veličiny na dané ose (včetně její
  jednotky). V případě grafu kombinujícího více jevů i legenda
  dokumentující grafické vyjádření jevů v grafu.
\item
  \textbf{veškeré tabulky a grafy musí být komentovány v textu - čtenáři
  musí řečeno, co v tom grafu uvidí a čeho si má všimnout.}
\end{itemize}






    \addcontentsline{toc}{section}{Reference}
    \begin{thebibliography}{zdroje}
        \bibitem{slajdy} \label{ims} PERINGER P. Slajdy k přednáškám do předmětu Modelování a simulace, 2017. Verze 15. září 2017 [cit. 2018-12-05]
        \bibitem{slajdy} \label{kniha} KOVAŘÍK, František. Hmyz: chov, morfologie. Jihlava: Madagaskar, 2000. ISBN 80-86068-24-2.
    \end{thebibliography}


\end{document}