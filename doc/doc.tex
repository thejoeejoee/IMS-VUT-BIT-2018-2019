\documentclass{article}
\documentclass[11pt, a4paper, titlepage]{article}

% Set document dimensions
\usepackage[paper=a4paper,top=1cm,left=2cm,right=2cm,bottom=2cm, includefoot]{geometry}
\usepackage{float}
\usepackage[final]{pdfpages} % inludesvg

% Czech fonts
\usepackage[T1]{fontenc}
\usepackage[utf8]{inputenc}
\usepackage[czech]{babel}
\usepackage{amsmath}
\usepackage{listings}
\usepackage{wrapfig}
% \usepackage{titlesec}
\usepackage{color} % barvy
\usepackage{xcolor} % vytváření barev
\usepackage{multicol}
\usepackage{newverbs}
\usepackage{lipsum}

%\usepackage{fancyhdr}

\usepackage{fancyhdr}


\lstnewenvironment{code}[1][]%
{\noindent\minipage{\linewidth}\lstset{frameround=fttf,#1}}%
{\endminipage}%

\definecolor{codeprimary}{HTML}{3300CC}
\colorlet{keywordstyle}{codeprimary!50!black}
\lstset{
language=C++,
frameround=fftf,
breaklines=true,
keywordstyle=\color{keywordstyle}\ttfamily,
basicstyle=\color{codeprimary},
numberstyle=\color{black},
backgroundcolor=\color{white},
frame=single,
tabsize=4,
breaklines=true,
captionpos=t,
xleftmargin=\dimexpr\fboxsep+1pt\relax,
xrightmargin=\fboxsep,
numbers=none,
showstringspaces=false,
escapeinside={\#!}{\^^M},
belowcaptionskip=0pt,
belowskip=0pt,
aboveskip=0pt,
}

\makeatletter
\newcommand\ic[1][green]{%
\@testopt{\@ic{#1}}{-#1}% Handle second optional argument
}
\def\@ic#1[#2]{%
\Collectverb{\@@ic{#1}{#2}}%
}
\def\@@ic#1#2#3{%
{\lstinline[basicstyle=\ttfamily\color{codeprimary},breaklines=true]|#3|}%
}
\newcommand{\icmacro}[1]{{\lstinline[basicstyle=\ttfamily\color{codeprimary},breaklines=true]|#1|}}
\makeatother


\setlength{\headheight}{3em}
\newcommand{\subsectionbreak}{\clearpage}

\begin{document}
    %%%%%%%%%%%%%%%%%%%%%%%%%%%%%%%%%%%%%%%%%%%%%%%%%%%%%%%%%%%%%%%%%%
% Dokumentační část projektu do IMS.
% VUT FIT
% @author Josef Kolář, xkolar71
% @author Iva Kavánková, xkavan05
% @date 2018; 05; 12
%%%%%%%%%%%%%%%%%%%%%%%%%%%%%%%%%%%%%%%%%%%%%%%%%%%%%%%%%%%%%%%%%%

\begin{titlepage}
    % \newgeometry{top=1in,top=2cm,left=2cm,right=2cm,bottom=2cm}

    \centering

    {\fontsize{20pt}{15pt}\bfseries
    VYSOKÉ UČENÍ TECHNICKÉ V~BRNĚ\\
    \vspace{8pt}
    Fakulta informačních technologií
    }

    \includegraphics[scale=0.7]{./assets/fit-logo.pdf}

    \vspace{22pt}

    {\Large Modelování a simulace\\}
    \vspace{4pt}
    {\LARGE \bfseries Chov hmyzu pro potravinářské a průmyslové účely}

    \vspace{180pt}
    {\Large \today}

    \vspace{90pt}
    {\Large \bfseries Autoři\\}
    \vspace{12pt}

    \begin{tabular}{ l c r }
        Josef Kolář & \texttt{xkolar71} \\
        Iva Kavánková & \texttt{xkavan05} \\
    \end{tabular}\\

\end{titlepage}
    \pagestyle{fancy}
    \lfoot{\emph{VUT FIT - IMS}}
    \rfoot{\emph{Josef Kolář, Iva Kavánková}}
    \rhead{...}
    \lhead{Modelování a simulace}
    \tableofcontents
    \pagebreak
    %\noindent\makebox[\linewidth]{\rule{\textwidth}{0.4pt}}

    \section{Úvod}
    Tato práce vznikla jako projekt do předmětu Modelování a simulace. Zabývá se simulací (viz [\ref{ims}], slajd č. 8)
    modelu (viz [\ref{ims}], slajd č. 7) chovu cvrčků banánových (dále jen crvčci). Na základě daného modelu a sady simulačních experimentů (viz [\ref{ims}], slajd č. 33),
    bude ukázáno chování systému v rozdílných podmínkách. Smyslem projektu je demonstrovat, smysluplnost a finanční efektivitu hypotetické farmy na cvrčky.

    \subsection{Zdroje informací}
    Autoři projektu jsou Josef Kolář a Iva Kavánková. Při tvorbě došlo k čeprání znalostí z přednášek předmětu Modelování a simulace, z odborné literatury
    [\ref{kniha}], napsání dotazů na odborné internetové fórum (viz [\ref{forum}]) a taktéž od inženýrky Olgy Kavánkové, která je středoškolskou pedagožkou s aprobací na biologii
    a matematiku, čímž ji patří velké poděkování. Díky jejím odborným znalostem byl vytvořen odpovídající abstraktní model
    (viz [\ref{ims}], slajd č. 9).

    \subsection{Ověřování validity modelu}
    Validita (viz [\ref{ims}], slajd č. 37) byla ověřována při našem postupném testování. Výsledky byly porovnány s dříve získanými daty.

    \section{Rozbor tématu a použitých metod/technologií}
    Pro modelování a simulaci hmyzí farmy je nutné znát její reálný chod. \\
    Životní cyklus cvrčka začíná jeho vylíhnutím, poté následuje doba dospívání,po které je jich schopen reprodukce a zároveň
    je také vhodný na prodej pro potravinářské účely. Samice po oplodnění samcem je schopna klást vajíčka, ze kterých vzniká
    nová populace. Nicméně, ne ze všech vajíček se narodí živí jedinci. Je také nutné brát v úvahu úbytek samců po soubojích, které provádí mezi sebou. Různě staré cvrčky je nutno chovat v generačních nádobách,
    neboť jinak dochází k velké míře kanibalismu. \\
    Shrnutí použitých číselných údajů znázorňuje následující tabulka.

    \begin{table}[h]
        \begin{tabular}{|c|c|c|}
        \hline
        \textbf{Popis}                                         & \textbf{Hodnota}      & \textbf{Zdroj}                                       \\ \hline
        Doba za kterou cvrček dospěje (od narození)            & 4 týdny               & Ing. Olga Kavánková                                  \\ \hline
        Doba za kterou je cvrček vhodný k prodeji              & 4 týdny               & Převazato z {[}\textbackslash{}ref\{kniha\}{]}       \\ \hline
        Počet nakladených vajec od jedné samice denně          & 10                    & Převazato z {[}\textbackslash{}ref\{kniha\}{]}       \\ \hline
        Procentuální úspěšnost vylíhnutí z vajíčka             & 90 \%                 & Ing. Olga Kavánková                                  \\ \hline
        Doba za kterou se cvrček vylíhne z nakladeného vajíčka & 2 týdny               & Převazato z {[}\textbackslash{}ref\{kniha\}{]}       \\ \hline
        Celková doba života cvrčka                             & 12 týdnů              & Ing. Olga Kavánková                                  \\ \hline
        Počet jedinců v 1 litru                                & 600 ks                & Převazato z {[}\textbackslash{}ref\{forum\}{]}       \\ \hline
        Cena 1 litru cvrčků                                    & 300 Kč                & Průměrná hodnota z {[}\textbackslash{}ref\{cena\}{]} \\ \hline
        Úbytek samců po soubojích (denně)                      & 5 \%                  & Ing. Olga Kavánková                                  \\ \hline
        Hmotnost jídla, co sní crvček starý 0 - 4 týdny        & 0,016 g               & Převazato z {[}\textbackslash{}ref\{jidlo\}{]}       \\ \hline
        Hmotnost jídla, co sní cvrček starší 4 týdnů           & 0,034 g               & Převazato z \textbackslash{}ref\{jidlo\}             \\ \hline
        \multicolumn{1}{|l|}{}                                 & \multicolumn{1}{l|}{} & \multicolumn{1}{l|}{}                                \\ \hline
        \multicolumn{1}{|l|}{}                                 & \multicolumn{1}{l|}{} & \multicolumn{1}{l|}{}                                \\ \hline
        \multicolumn{1}{|l|}{}                                 & \multicolumn{1}{l|}{} & \multicolumn{1}{l|}{}                                \\ \hline
        \end{tabular}
    \end{table}



    \subsection{Použité postupy}
    Program byl implementován v jazyce C++ s pomocí simulační knihovny SIMLIB (viz [\ref{simlib}]). To usnadňuje objektový návrh
    a třídy vhodné pro simulaci konkrétního zadání. Použité algoritmy byly použity ze slajdů k předmětu IMS (viz [\ref{ims}]), slajdy
    z prvního (viz [\ref{prvnidemo}]) a druhého (viz [\ref{druhedemo}]) demonstračního cvičení z tohoto předmětu.

    \\

    \section{Koncepce modelu}

    \subsection{Návrh konceptuálního modelu}

    \subsection{Formy konceptuálního modelu}

    \section{Architektura simulačního modelu}

    Jádro simulační modelu leží ve třídách \ic|Cricket| a \ic|HatchEvent|. V případě první se jedná o proces modelující
    životní cyklus jednoho cvrčka, jakožto jedince. Druhá třída poté simuluje pomocí události nakladené vajíčko cvrčka -
    její naplánování do kalendáře zajištují samičky.

    \subsection{Rozbor implementace}

    \section{Podstata simulačních experimentů a jejich průběh}

    \subsection{Postup experimentování}

    \subsection{Dokumentace jednotlivých experimentů}

    \subsection{Závěry experimentů}

    \section{Shrnutí simulačních experimentů a závěr}





    \addcontentsline{toc}{section}{Reference}
    \begin{thebibliography}{zdroje}
        \bibitem{1} \label{ims} PERINGER P. Slajdy k přednáškám do předmětu Modelování a simulace, 2017. Verze 15. září 2017 [cit. 2018-12-05]
        \bibitem{2} \label{kniha} KOVAŘÍK, František. Hmyz: chov, morfologie. Jihlava: Madagaskar, 2000. ISBN 80-86068-24-2.
        \bibitem{3} \label{simlib} PERINGER P. Simulační knihovna SIMLIB pro C++, dostupné na https://www.fit.vutbr.cz/~peringer/SIMLIB/
        \bibitem{4} \label{prvnidemo} HRUBÝ M. Slajdy k I. democvičení, 2011 [cit. 2018-12-05]
        \bibitem{5} \label{druhedemo} HRUBÝ M. Slajdy k II. democvičení, 2011 [cit. 2018-12-05]
        \bibitem{6} \label{jidlo} [Online] http://www.openbugfarm.com/forum.html?fbclid=IwAR3tl0-KFHtvkUVEI5XdTn54CVVGcY5vnApXX0xmOnOQ5B78kurOgmtlcwo#/discussion/1111/how-much-do-crickets-eat [cit. 2018-12-05]
        \bibitem{7} \label{forum} [Online] https://www.ifauna.cz/terarijni-zvirata/diskuse/detail/3524112/chov-cvrcku-v-cislech#a3524888 [cit. 2018-12-05]
        \bibitem{7} \label{cena} [Online] https://www.broukarna.cz/cvrcci/cvrcek-bananovy-2-2/ [cit. 2018-12-05]
    \end{thebibliography}


\end{document}