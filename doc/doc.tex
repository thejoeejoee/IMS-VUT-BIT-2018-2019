\documentclass{article}
\documentclass[11pt, a4paper, titlepage]{article}

% Set document dimensions
\usepackage[paper=a4paper,top=1cm,left=2cm,right=2cm,bottom=2cm, includefoot]{geometry}
\usepackage{float}
\usepackage[final]{pdfpages} % inludesvg

% Czech fonts
\usepackage[T1]{fontenc}
\usepackage[utf8]{inputenc}
\usepackage[czech]{babel}
\usepackage{amsmath}
\usepackage{listings}
\usepackage{wrapfig}
% \usepackage{titlesec}
\usepackage{color} % barvy
\usepackage{xcolor} % vytváření barev
\usepackage{multicol}
\usepackage{newverbs}
\usepackage{lipsum}

%\usepackage{fancyhdr}

\usepackage{fancyhdr}


\lstnewenvironment{code}[1][]%
{\noindent\minipage{\linewidth}\lstset{frameround=fttf,#1}}%
{\endminipage}%

\definecolor{codeprimary}{HTML}{3300CC}
\colorlet{keywordstyle}{codeprimary!50!black}
\lstset{
    language=C++,
    frameround=fftf,
    breaklines=true,
    keywordstyle=\color{keywordstyle}\ttfamily,
    basicstyle=\color{codeprimary},
    numberstyle=\color{black},
    backgroundcolor=\color{white},
    frame=single,
    tabsize=4,
    breaklines=true,
    captionpos=t,
    xleftmargin=\dimexpr\fboxsep+1pt\relax,
    xrightmargin=\fboxsep,
    numbers=none,
    showstringspaces=false,
    escapeinside={\#!}{\^^M},
    belowcaptionskip=0pt,
    belowskip=0pt,
    aboveskip=0pt,
}

\makeatletter
\newcommand\ic[1][green]{%
\@testopt{\@ic{#1}}{-#1}% Handle second optional argument
}
\def\@ic#1[#2]{%
\Collectverb{\@@ic{#1}{#2}}%
}
\def\@@ic#1#2#3{%
{\lstinline[basicstyle=\ttfamily\color{codeprimary},breaklines=true]|#3|}%
}
\newcommand{\icmacro}[1]{{\lstinline[basicstyle=\ttfamily\color{codeprimary},breaklines=true]|#1|}}
\makeatother


\setlength{\headheight}{3em}
\newcommand{\subsectionbreak}{\clearpage}

\begin{document}
    %%%%%%%%%%%%%%%%%%%%%%%%%%%%%%%%%%%%%%%%%%%%%%%%%%%%%%%%%%%%%%%%%%
% Dokumentační část projektu do IMS.
% VUT FIT
% @author Josef Kolář, xkolar71
% @author Iva Kavánková, xkavan05
% @date 2018; 05; 12
%%%%%%%%%%%%%%%%%%%%%%%%%%%%%%%%%%%%%%%%%%%%%%%%%%%%%%%%%%%%%%%%%%

\begin{titlepage}
    % \newgeometry{top=1in,top=2cm,left=2cm,right=2cm,bottom=2cm}

    \centering

    {\fontsize{20pt}{15pt}\bfseries
    VYSOKÉ UČENÍ TECHNICKÉ V~BRNĚ\\
    \vspace{8pt}
    Fakulta informačních technologií
    }

    \includegraphics[scale=0.7]{./assets/fit-logo.pdf}

    \vspace{22pt}

    {\Large Modelování a simulace\\}
    \vspace{4pt}
    {\LARGE \bfseries Chov hmyzu pro potravinářské a průmyslové účely}

    \vspace{180pt}
    {\Large \today}

    \vspace{90pt}
    {\Large \bfseries Autoři\\}
    \vspace{12pt}

    \begin{tabular}{ l c r }
        Josef Kolář & \texttt{xkolar71} \\
        Iva Kavánková & \texttt{xkavan05} \\
    \end{tabular}\\

\end{titlepage}
    \pagestyle{fancy}
    \lfoot{\emph{VUT FIT - IMS}}
    \rfoot{\emph{Josef Kolář, Iva Kavánková}}
    \rhead{...}
    \lhead{Modelování a simulace}
    \tableofcontents
    \pagebreak
    %\noindent\makebox[\linewidth]{\rule{\textwidth}{0.4pt}}

    \section{Úvod}
    Tato práce vznikla jako projekt do předmětu Modelování a simulace. Zabývá se simulací (viz [\ref{ims}], slajd č. 8)
    modelu (viz [\ref{ims}], slajd č. 7) chovu cvrčků banánových (dále jen crvčci). Na základě daného modelu a sady simulačních experimentů (viz [\ref{ims}], slajd č. 33),
    bude ukázáno chování systému v rozdílných podmínkách. Smyslem projektu je demonstrovat, smysluplnost a finanční efektivitu hypotetické farmy na cvrčky.

    \subsection{Zdroje informací}
    Autoři projektu jsou Josef Kolář a Iva Kavánková. Při tvorbě došlo k čeprání znalostí z přednášek předmětu Modelování a simulace, z odborné literatury
    [\ref{kniha}] a obzvlášť od paní inženýrky Olgy Kavánkové, která je středoškolskou pedagožkou s aprobací na biologii
    a matematiku, čímž ji patří velké poděkování. Díky jejím odborným znalostem byl vytvořen odpovídající abstraktní model
    (viz [\ref{ims}], slajd č. 9).

    \subsection{Ověřování validity modelu}
    Validita (viz [\ref{ims}], slajd č. 37) byla ověřována při našem postupném testování. Výsledky byly porovnány s dříve získanými daty.

    \section{Rozbor tématu a použitých metod/technologií}
    Pro modelování a simulaci hmyzí farmy je nutné znát její reálný chod. \\
    Životní cyklus cvrčka začíná jeho vylíhnutím, poté následuje

     \subsection{Použité postupy}
    Program byl implementován v jazyce C++ s pomocí simulační knihovny SIMLIB (viz [\ref{simlib}]). To usnadňuje objektový návrh
    a třídy vhodné por simulaci konkrétního zadání. Použité algoritmy byly použity ze slajdů k předmětu IMS (viz [\ref{ims}]), slajdy
    z prvního (viz [\ref{prvnidemo}]) a druhého (viz [\ref{druhedemo}]) demonstračního cvičení z tohoto předmětu.

    \\

    \section{Koncepce modelu}

    \subsection{Návrh konceptuálního modelu}

    \subsection{Formy konceptuálního modelu}

    \section{Architektura simulačního modelu}

    \subsection{Rozbor implementace}

    \section{Podstata simulačních experimentů a jejich průběh}

    \subsection{Postup experimentování}

    \subsection{Dokumentace jednotlivých experimentů}

    \subsection{Závěry experimentů}

    \section{Shrnutí simulačních experimentů a závěr}





    \addcontentsline{toc}{section}{Reference}
    \begin{thebibliography}{zdroje}
        \bibitem{1} \label{ims} PERINGER P. Slajdy k přednáškám do předmětu Modelování a simulace, 2017. Verze 15. září 2017 [cit. 2018-12-05]
        \bibitem{2} \label{kniha} KOVAŘÍK, František. Hmyz: chov, morfologie. Jihlava: Madagaskar, 2000. ISBN 80-86068-24-2.
        \bibitem{3} \label{simlib} PERINGER P. Simulační knihovna SIMLIB pro C++, dostupné na https://www.fit.vutbr.cz/~peringer/SIMLIB/
        \bibitem{4} \label{prvnidemo} HRUBÝ M. Slajdy k I. democvičení, 2011 [cit. 2018-12-05]
        \bibitem{5} \label{druhedemo} HRUBÝ M. Slajdy k II. democvičení, 2011 [cit. 2018-12-05]
    \end{thebibliography}


\end{document}